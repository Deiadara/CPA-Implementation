\chapter{Experimental Evaluation}
\label{chap:experimental}

This chapter presents our experimental evaluation of the CPA protocol family: CPA, $\sigma$-CPA (CPA with dealer signatures), DS-CPA (Dolev-Strong CPA), and B-CPA (Bracha's CPA). We implemented all protocols in Python and conducted systematic benchmarks to compare their resilience under different corruption models.

\section{Experimental Setup}

\subsection{Implementation Details}

Our implementation models the message-passing environment where:
\begin{itemize}
    \item Each node maintains local state and communicates with neighbors
    \item Byzantine nodes can equivocate (send different values to different neighbors), withhold messages, and spam invalid messages
    \item The adversary uses $t(u)$-local corruption: each node $u$ has at most $t(u)$ Byzantine neighbors
    \item Cryptographic signatures use Ed25519 for authenticity guarantees
\end{itemize}

\subsection{Protocol Models}

The protocols operate under different timing models:

\textbf{Synchronous protocols} (CPA, $\sigma$-CPA, DS-CPA):
\begin{itemize}
    \item \textbf{CPA}: $n$ rounds
    \item \textbf{$\sigma$-CPA}: $n$ rounds (same as CPA, with dealer signatures)
    \item \textbf{DS-CPA}: $(f+1) \times n$ rounds (Dolev-Strong requires $f+1$ signature chain rounds)
\end{itemize}

\textbf{Asynchronous protocol} (B-CPA):
\begin{itemize}
    \item B-CPA operates in an \emph{asynchronous} model with no round structure
    \item We simulate $4n$ message-passing ``steps'' but these are \emph{not comparable} to synchronous rounds
    \item B-CPA relies on quorum intersection ($n \geq 3f+1$) rather than timing guarantees
    \item \textbf{Termination is NOT guaranteed} with dishonest dealer---this is expected behavior
\end{itemize}

\subsection{Byzantine Reliable Broadcast Properties}

The protocols aim to satisfy three properties:
\begin{itemize}
    \item \textbf{Termination}: Every honest node eventually decides on some value
    \item \textbf{Agreement}: All honest nodes that decide, decide on the same value
    \item \textbf{Validity}: If the dealer is honest, every honest node decides on the dealer's value
\end{itemize}

In the asynchronous model, \emph{Termination cannot be guaranteed} when the dealer is dishonest (due to FLP impossibility). B-CPA prioritizes \emph{Agreement}---if nodes decide, they agree. Not deciding is considered safe behavior.

\section{Benchmark Methodology}

For each experiment, we sample a graph and corruption set \emph{once}, then run \emph{all protocols} on the same configuration. This ensures fair comparison: identical Byzantine nodes for every algorithm. We ran 100 samples per graph type across 5 different topologies (500 total samples per protocol).

\section{Results}

\subsection{Specific Scenarios}

Table~\ref{tab:scenarios} shows protocol behavior under curated scenarios.

\begin{table}[htbp]
\centering
\caption{Protocol Comparison Under Specific Scenarios}
\label{tab:scenarios}
\small
\begin{tabular}{llccccc}
\hline
\textbf{Scenario} & \textbf{Protocol} & \textbf{Success} & \textbf{Rnds/Steps} & \textbf{Decided} & \textbf{Dealer Byz} \\
\hline
All Succeed & CPA & \checkmark & 10 & 9/9 & No \\
(Complete) & $\sigma$-CPA & \checkmark & 10 & 9/9 & No \\
 & DS-CPA & \checkmark & 90 & 9/9 & No \\
 & B-CPA$^*$ & \checkmark & 40 & 9/9 & No \\
\hline
$\sigma$-CPA & CPA & $\times$ & 10 & 5/9 & No \\
Advantage & $\sigma$-CPA & \checkmark & 10 & 9/9 & No \\
 & DS-CPA & \checkmark & 90 & 9/9 & No \\
 & B-CPA$^*$ & $\times$ & 40 & 0/9 & No \\
\hline
Dishonest & DS-CPA & \checkmark & 90 & 9/9 & Yes \\
Dealer & B-CPA$^*$ & \checkmark$^\dagger$ & 40 & 0/8 & Yes \\
\hline
\multicolumn{6}{l}{\footnotesize $^*$B-CPA steps are not comparable to synchronous rounds} \\
\multicolumn{6}{l}{\footnotesize $^\dagger$Agreement holds vacuously (0 decided $\Rightarrow$ no disagreement)}
\end{tabular}
\end{table}

\textbf{Key observation on B-CPA with dishonest dealer:} When the dealer equivocates (sends different values to different nodes), honest nodes may not reach the $n-f$ quorum threshold for ECHO/VOTE messages. In this case, they do not decide---which is \emph{safe} behavior. Agreement holds vacuously: with 0 nodes deciding, there can be no disagreement. This contrasts with DS-CPA, which is synchronous and guarantees termination even with dishonest dealer.

\subsection{Synchronous Protocol Results}

Table~\ref{tab:sync-stats} shows results for the synchronous protocols.

\begin{table}[htbp]
\centering
\caption{Synchronous Protocol Performance (100 samples per graph, same corruption set)}
\label{tab:sync-stats}
\small
\begin{tabular}{llcccc}
\hline
\textbf{Graph} & \textbf{Protocol} & \textbf{Success \%} & \textbf{Rounds} & \textbf{Avg Actual} & \textbf{Coverage \%} \\
\hline
Complete ($n$=15) & CPA & 100.0\% & 15 & 1.0 & 100.0\% \\
 & $\sigma$-CPA & 100.0\% & 15 & 1.0 & 100.0\% \\
 & DS-CPA & 100.0\% & 210 & 210.0 & 100.0\% \\
\hline
Dense Random ($n$=15) & CPA & 100.0\% & 15 & 2.0 & 100.0\% \\
 & $\sigma$-CPA & 100.0\% & 15 & 2.0 & 100.0\% \\
 & DS-CPA & 100.0\% & 210 & 210.0 & 100.0\% \\
\hline
Random Regular ($n$=15) & CPA & 97.0\% & 15 & 4.8 & 98.8\% \\
 & $\sigma$-CPA & 100.0\% & 15 & 3.0 & 100.0\% \\
 & DS-CPA & 100.0\% & 210 & 210.0 & 100.0\% \\
\hline
Cycle ($n$=15) & CPA & 0.0\% & 15 & 15.0 & 24.6\% \\
 & $\sigma$-CPA & 9.0\% & 15 & 14.5 & 52.4\% \\
 & DS-CPA & 9.0\% & 210 & 210.0 & 52.4\% \\
\hline
Hypercube ($n$=16) & CPA & 100.0\% & 16 & 4.0 & 100.0\% \\
 & $\sigma$-CPA & 100.0\% & 16 & 3.9 & 100.0\% \\
 & DS-CPA & 100.0\% & 240 & 240.0 & 100.0\% \\
\hline
\end{tabular}
\end{table}

\subsection{B-CPA Results (Asynchronous)}

Table~\ref{tab:bcpa-stats} shows B-CPA results separately. B-CPA's success rate reflects whether Agreement holds (including the case of 0 decisions).

\begin{table}[htbp]
\centering
\caption{B-CPA Performance (Asynchronous Protocol)}
\label{tab:bcpa-stats}
\small
\begin{tabular}{lccc}
\hline
\textbf{Graph} & \textbf{Success \%} & \textbf{Steps ($4n$)} & \textbf{Coverage \%} \\
\hline
Complete ($n$=15) & 100.0\% & 60 & 100.0\% \\
Dense Random ($n$=15) & 100.0\% & 60 & 100.0\% \\
Random Regular ($n$=15) & 7.0\% & 60 & 12.2\% \\
Cycle ($n$=15) & 0.0\% & 60 & 0.0\% \\
Hypercube ($n$=16) & 9.0\% & 64 & 18.7\% \\
\hline
\end{tabular}
\end{table}

B-CPA's low success on sparse graphs reflects its quorum requirement: nodes must collect $n-f$ ECHO and VOTE messages. On graphs where nodes have fewer neighbors than $n-f$, quorum cannot be reached.

\subsection{CPA vs $\sigma$-CPA: Head-to-Head}

Table~\ref{tab:cpa-sigma-comparison} directly compares CPA and $\sigma$-CPA on identical corruption sets.

\begin{table}[htbp]
\centering
\caption{CPA vs $\sigma$-CPA: Same Corruption Set Comparison}
\label{tab:cpa-sigma-comparison}
\small
\begin{tabular}{lcccc}
\hline
\textbf{Graph} & \textbf{Both Succeed} & \textbf{$\sigma$-CPA Only} & \textbf{CPA Only} & \textbf{Neither} \\
\hline
Complete ($n$=15) & 100 & 0 & 0 & 0 \\
Dense Random ($n$=15) & 100 & 0 & 0 & 0 \\
Random Regular ($n$=15) & 97 & 3 & 0 & 0 \\
Cycle ($n$=15) & 0 & 9 & 0 & 91 \\
Hypercube ($n$=16) & 100 & 0 & 0 & 0 \\
\hline
\textbf{Total (500 samples)} & 397 & 12 & 0 & 91 \\
\hline
\end{tabular}
\end{table}

The ``CPA Only'' column is always 0: $\sigma$-CPA \emph{never performs worse} than CPA.

\subsection{Aggregate Results}

Table~\ref{tab:aggregate} summarizes synchronous protocol performance across all 500 samples.

\begin{table}[htbp]
\centering
\caption{Aggregate Synchronous Protocol Performance (500 samples)}
\label{tab:aggregate}
\small
\begin{tabular}{lccc}
\hline
\textbf{Protocol} & \textbf{Success \%} & \textbf{Avg Actual Rounds} & \textbf{Avg Coverage \%} \\
\hline
CPA & 79.4\% & 5.4 & 84.7\% \\
$\sigma$-CPA & 81.8\% & 4.9 & 90.5\% \\
DS-CPA & 81.8\% & 216.0 & 90.5\% \\
\hline
\end{tabular}
\end{table}

\section{Analysis}

\subsection{Synchronous vs Asynchronous Trade-offs}

\begin{center}
\begin{tabular}{lcccc}
\hline
\textbf{Protocol} & \textbf{Model} & \textbf{Honest Dealer} & \textbf{Dishonest Dealer} & \textbf{Rounds} \\
\hline
CPA & Sync & Termination + Validity & Fails & $n$ \\
$\sigma$-CPA & Sync & Termination + Validity & Fails & $n$ \\
DS-CPA & Sync & Termination + Validity & Agreement + Termination & $(f+1) \times n$ \\
B-CPA & Async & Termination + Validity & Agreement (no termination) & N/A \\
\hline
\end{tabular}
\end{center}

DS-CPA guarantees both Agreement AND Termination with dishonest dealer, at the cost of $(f+1) \times n$ synchronous rounds. B-CPA only guarantees Agreement (may not decide), but operates asynchronously.

\subsection{Resilience Hierarchy}

\begin{enumerate}
    \item \textbf{$\sigma$-CPA $\geq$ CPA}: Dealer signatures provide strictly better resilience.
    
    \item \textbf{DS-CPA matches $\sigma$-CPA resilience}: Same success rate, extra rounds for dishonest dealer tolerance.
    
    \item \textbf{B-CPA requires dense graphs}: Quorum-based design needs near-complete connectivity.
\end{enumerate}

\subsection{Topology Sensitivity}

\begin{itemize}
    \item \textbf{Complete/Dense}: All protocols perform well (100\% success)
    \item \textbf{Hypercube}: Good connectivity ($\log n$ degree) enables CPA/$\sigma$-CPA success
    \item \textbf{Cycle}: Minimal connectivity causes widespread failures
\end{itemize}

\section{Conclusions}

\begin{enumerate}
    \item \textbf{$\sigma$-CPA dominates CPA}: Same round complexity, strictly better resilience.
    
    \item \textbf{DS-CPA provides full dishonest dealer tolerance}: Guarantees Agreement AND Termination at cost of $(f+1)\times n$ rounds.
    
    \item \textbf{B-CPA is asynchronous with safety guarantees}: May not terminate with dishonest dealer, but Agreement always holds. Requires dense networks.
    
    \item \textbf{Topology matters}: Sparse graphs limit all protocols; B-CPA especially requires high connectivity.
\end{enumerate}

For practitioners: use $\sigma$-CPA when dealer honesty is assured, DS-CPA when dishonest dealer must be tolerated with termination guarantees, and B-CPA only on dense networks in asynchronous settings.
